\chapter{Conclusions and Future Work}
\section{Conclusions}\label{sec-concl}

The key to the applicability of the GMCR is that it contains both modeling and analysis components. Many real-world conflict decision problems exhibit preference uncertainty, but until now there was no suitable methodology to model and analyze them. This thesis has remedied this problem.

Together with the non-cooperative form of fuzzy stabilities, the concepts of coalitional form of fuzzy stabilities are put forward to provide follow-up analysis of a dispute. Addressing the difficulties of modeling fuzzy preferences from uncertain preference information, the fuzzy option prioritization methodology is developed within the framework of the GMCR to efficiently model fuzzy preferences of DMs for use in the graph model fuzzy stability analysis. The main contributions of this study are summarized below:

\begin{enumerate}
\item A fuzzy preference framework for a two-DM graph model is developed to introduce fuzzy stability concepts and apply them to simple conflicts with two DMs who have uncertain preferences over feasible states. Specifically,
  \begin{itemize}
    \item The concept of an FRCP is introduced to characterize preference intensity between two feasible states in the presence of preference uncertainty (Section \ref{sec-frsp}).
    \item A parameter called ``FST" is introduced to take into account the interacting DMs' satisficing behavior in strategic conflicts, and is incorporated into the graph model fuzzy stability definitions (Section \ref{sec-fst}).
    \item To identify states to which a DM would be willing to move to, the concept of an FUI for a DM is put forward (Section \ref{sec-fui-dm}).
    \item The four basic crisp stability definitions for a two-DM graph model---\emph{R}, \emph{GMR}, \emph{SMR}, and \emph{SEQ}---are generalized as \emph{FR}, \emph{FGMR}, \emph{FSMR}, and \emph{FSEQ} to facilitate the (fuzzy) stability analysis of a two-DM graph model with uncertain preference information (Section \ref{sec-fuz-stabl-2dm}).
    \item The fuzzy stability definitions for a two-DM graph model are illustrated using the sustainable development conflict. It is found that the developers' satisficing behavior greatly influences the fuzzy stability results, which is quite reasonable because developers' decisions in development activities are volatile and mostly context-dependent. Environmental agencies' roles in choosing how strictly to enforce environmental regulations may guide developers' satisficing behavior. If developers want to grab every last penny, environmental security will remain under threat and environmental disasters can never be completely avoided (Section \ref{sec-appl-sus-dev}).
  \end{itemize}

\item The fuzzy preference framework for the two-DM graph model is extended to accommodate graph models with any number of DMs, generalizing the fuzzy stability definitions. More specifically,
  \begin{itemize}
    \item The concept of an FUI by a group or coalition of DMs is introduced for use in the fuzzy stabilities (Section \ref{sec-fui-coal}).
    \item  \emph{FGMR}, \emph{FSMR}, and \emph{FSEQ} stability definitions for a two-DM graph model are extended for a general $n$-DM ($n\geq 2$) graph model. In these definitions, fuzzy stabilities for a DM depend on the responses (credible or not) of the coalition of the remaining DMs, rather than the single opponent DM of the two-DM case. Since \emph{FR} stability does not depend on the responses by the opponent(s), the \emph{FR} stability definition for a general $n$-DM graph model remains the same as in the two-DM case (Section \ref{sec-fuz-stabl-ndm}).
    \item A fuzzy stability analysis is carried out on the Elmira groundwater contamination conflict by employing the fuzzy stability definitions for an $n$-DM graph model. This analysis predicts a new strong equilibrium (state $s_4$), indicated in Table \ref{tbl-comparison}, that is a possible resolution under all four fuzzy stability definitions. The fuzzy stability results tableau provided in Table \ref{tbl-fsa} demonstrates how a DM's satisficing criteria can affect the final outcome (Section \ref{sec-stabl-appl-elmira}).
    \item The FGM is shown to be a more general approach to decision making under conflict compared to the crisp graph model, as it can handle both certain and uncertain (fuzzy) preferences. Hence, the FGM constitutes an extension of the crisp graph model that permits the modeling and analysis of more realistic multiple participant-multiple objective decision problems. In particular, by setting the FST of each DM to $1.0$, FGM becomes the crisp GMCR (Remarks \ref{rmk-fui-dm}, \ref{rmk-fui-coal}, and Theorem \ref{thm-cgm-sc-fgm}).
    \item Like the crisp GMCR in which the associated preferences can be transitive or nontransitive, the FGM is developed such that any transitive or nontransitive fuzzy preferences can be utilized in the fuzzy stability calculations (Chapters \ref{chap-fuz-pref-2dm} and \ref{chap-fp-ndm-gm}).
  \end{itemize}

\item The coalition fuzzy stability concept is developed as a follow-up analysis technique within the FGM, intending to further analyze the individual level fuzzy stabilities of a conflict model with uncertain preference information. In particular,
  \begin{itemize}
    \item The concepts of CFI and CCFI are introduced as tools for facilitating the coalition fuzzy stability definitions (Section \ref{sec-fi-coals}).
    \item The coalition fuzzy stability definitions for \emph{R}, \emph{GMR}, \emph{SMR}, and \emph{SEQ} stability concepts of the GMCR are developed so that they constitute a natural generalization of the individual level \emph{FR}, \emph{FGMR}, \emph{FSMR} and \emph{FSEQ} stabilities (Section \ref{sec-cfs}).
    \item Coalition fuzzy stability definitions can be applied to a crisp graph model by assigning an FST of $1.0$ to each DM, thereby making them more general coalition analysis tools within the graph model structure. Accordingly, the four coalition fuzzy stability definitions---\emph{CFR}, \emph{CFGMR}, \emph{CFSMR}, and \emph{CFSEQ}---form a strong solution methodology for strategic conflicts with both certain and uncertain preference information. Although the implementation of these stabilities is not straightforward, a suitable decision support system could bring this capability to the fingertips of the DM and the analyst (Section \ref{sec-cfs}).
    \item A coalition fuzzy stability definition for a DM identifies states from which neither the DM himself or herself, nor any of the coalitions that he or she can join, would like to move away. These characteristics are justified regarding a particular DM, M, when the coalition fuzzy stability definitions are applied to the Elmira groundwater contamination conflict. Specifically, some states fail to be coalition fuzzy stable for M that were fuzzy stable with respect to non-cooperative fuzzy stability definitions. Accordingly, the coalition fuzzy stability analysis may narrow down the list of individual-level fuzzy stabilities, thereby providing the analyst with valuable strategic insights into the conflict under study (Section \ref{sec-cfs-appl}).
    \item It can also be concluded from the application that the possible evolution of a conflict from a status quo state to a final outcome can be conveniently explained using CFILs. Therefore, as an analysis tool to augment individual-level fuzzy stabilities, coalition fuzzy stability analysis constitutes an important addition to the FGM (Section \ref{sec-cfs-appl}).
  \end{itemize}

\item The fuzzy option prioritization methodology is developed within the FGM structure to facilitate the modeling of fuzzy preferences for DMs involved in a strategic conflict with uncertain preference information. Specifically,
  \begin{itemize}
    \item Fuzzy option prioritization is the first formal methodology to model a fuzzy preference within the GMCR framework in order to deal with uncertain preferences. This technique offers flexibility to DMs or analysts who can assume the intensity of truth of a preference statement at a feasible state to be any number between $0$ and $1$, referred to as a fuzzy truth value (Chapter \ref{chap-fop}).
    \item A fuzzy preference relation over the set of feasible states is constructed by taking into account the fuzzy truth values of preference statements at feasible states (Sections \ref{sec-ftv-fs} and \ref{sec-fpe}).
    \item The fuzzy option prioritization methodology generalizes the existing crisp option prioritization technique in the sense that crisp option prioritization is a special case of fuzzy option prioritization (Theorem \ref{thm-cop-sc-fop}).
    \item When applied to the Elmira groundwater contamination conflict, the methodology models fuzzy preferences for the DMs efficiently so that they are close to those obtained by a complicated human assessment based on pairwise comparison of states (Section \ref{sec-appl-fop}).
    \item For larger problems, modeling fuzzy preferences by pairwise comparison of states is unrealistic and may be impossible. However, the fuzzy option prioritization methodology can be applied to a dispute of any size without difficulty (Chapter \ref{chap-fop}).
    \item Since the FGM is developed in this research to study multiple participant-multiple objective decision problems by carrying out fuzzy stability analysis and coalition fuzzy stability analysis based on DMs' fuzzy preferences, the introduction of the fuzzy option prioritization methodology, an efficient tool to model fuzzy preferences, will make the FGM more useful (Chapters \ref{chap-fuz-pref-2dm}, \ref{chap-fp-ndm-gm}, \ref{chap-cfsa}, and \ref{chap-fop}).
\end{itemize}
\end{enumerate}

\section{Future Work}\label{sec-fw}

The FGM methodology developed in this PhD thesis is a complete fuzzy preference approach for both modeling and analyzing real-world multiple participant-multiple objective decision problems with certain or uncertain preference information for DMs. However, fuzzy stability, introduced in this thesis, is a new concept and may therefore be integrated with recent developments and initiatives within the framework of the GMCR. A number of directions for potential future research is listed below.

\begin{itemize}
    \item \emph{Matrix Representations of the Fuzzy Stability Definitions}: A recent addition to the GMCR is a matrix representation of the graph model solution concepts for easy computer coding and manipulation \citep{Xu-et-al2009a, Xu-et-al2009b, Xu-et-al2011}. This idea may be adapted to FGM to represent fuzzy stability definitions.
  \item \emph{Fuzzy Stabilities with Transitive Fuzzy Preferences}: The fuzzy stability definitions proposed in this thesis are based on fuzzy preferences that are not restricted by any transitivity property. However, various transitivity properties may be imposed on fuzzy preferences to study their implications for fuzzy stability.
  \item \emph{Fuzzy Status Quo Analysis}: Status quo analysis technique \citep{Li-et-al2004b, Li-et-al2005a, Li-et-al2005b} was developed within the crisp GMCR to inspect whether a potential resolution (i.e., an equilibrium state) is attainable from the status quo state, and to analyze how DMs may act and interact to direct a conflict to that attainable resolution. A fuzzy version of the status quo analysis technique may be developed within the FGM to keep track of the evolution of a conflict from the status quo state to a \emph{FE}.
  \item \emph{Fuzzy Stability Definitions for Other Stability Types}: The fuzzy stability definitions introduced in this study are \emph{FR}, \emph{FGMR}, \emph{FSMR}, and \emph{FSEQ}. However, fuzzy stability definitions for other graph model stability concepts, such as \emph{limited move} and \emph{non-myopic} stabilities, may be defined by imposing appropriate transitivity property on DMs' fuzzy preferences.
  \item \emph{Use of Other Techniques to Model Fuzzy Preferences}: Fuzzy option prioritization, developed in this thesis, is the only methodology to formally model fuzzy preferences for DMs with uncertain preference information in a graph model, based on fuzzy truth values of preference statements at feasible states. Other techniques may be developed to represent DMs' fuzzy preferences by taking into account other uncertain information about preferences, such as fuzzy utilities, fuzzy option weighting, and fuzzy multi-criteria decision making.
  \item \emph{Use of Other Types of Fuzzy Preferences}: There may be a fuzzy preference relation in which the additive reciprocity condition is not met; that is, there may be a certain degree for a pair of states to which a DM does not prefer one state of the pair over the other. Techniques may be developed within the GMCR framework to handle this type of fuzzy preference relation.
  \item \emph{Decision Support System for} FGM: Calculating various graph model fuzzy stabilities by hand is tedious even for a small model. Therefore, the design of a suitable decision support system for the FGM may be an important future project.
  \item \emph{Applications to Challenging Real-World Conflicts}: FGM can be applied to many challenging real-world disputes to gain strategic insights.
\end{itemize}

%Fuzzy stability is a new concept and may be integrated with recent developments and initiatives within the framework of the Graph Model for Conflict Resolution, such as coalition analysis \cite{Kilgour-et-al01, Inohara&Hipel08a, Inohara&Hipel08b} and status quo analysis \cite{Li-et-al04, Li-et-al05, Li-et-al06}. The present fuzzy stability analysis prescribes what a DM can do when acting independently in a conflict, based on his or her own interests. Appropriate definitions for coalition fuzzy stabilities would identify advantageous outcomes for coalitions. A status quo analysis within the Fuzzy Preference Framework for the Graph Model may be carried out, to keep track of the evolution of a conflict from the status quo state. Furthermore, procedures and techniques should be developed in future research for conveniently eliciting fuzzy preference information from DMs involved in a given dispute. Calculating various Graph Model fuzzy stabilities by hand is tedious even for a small model, making the design of a suitable decision support system for the Fuzzy Preference Framework for the Graph Model an important future project.
